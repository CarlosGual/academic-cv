%-------------------------------------------------------------------------------
%	SECTION TITLE
%-------------------------------------------------------------------------------
\cvsection{Research Experience}


%-------------------------------------------------------------------------------
%	CONTENT
%-------------------------------------------------------------------------------
\begin{cventries}

%---------------------------------------------------------
    %---------------------------------------------------------
    \cventry
    {Advisor: Dr. Roberto Javier López Sastre} % Advisor
    {University of Alcalá - Dept of Telecommunications and Signal Theory} % Institution
    {Alcalá de Henares, Spain} % Location
    {Sep. 2021 - Present} % Date(s)
    {
        \begin{cvitems} % Description(s)
            \item {Development of reinforcement learning agents for visual semantic navigation in indoor environments using PyTorch.}
            \item {Implementation of artificial intelligence solutions over robotic platforms using ROS.}
            \item {Use of Gymnasium (former OpenAI Gym), Habitat, AI2-THOR, Isaac Lab and VizDoom simulators for training reinforcement learning agents.}
            \item {Exploration of several reinforcement learning algorithms: policy gradient methods (PPO, DD-PPO, TRPO), Q-learning methods (DQN, DDPG) and actor-critic methods (A2C, A3C) to resolve the usecase of visual semantic navigation.}
            \item {Exploration of meta reinforcement learning and meta imitation learning solutions for the sim-to-real gap in real robots.}
            \item {Integration of robotic platforms to help children with functional diversity.}
            \item {Use of Embodied AI techniques to improve the interaction between robots and its environment.}
            \item {Tuning of hyperparameters via Wandb Sweeps and exploration of the exploration-exploitation trade-off in reinforcement learning.}
            \item {Participation in two Knowledge Generation Projects funded by the Spanish Ministry of Science and Innovation.}
        \end{cvitems}
    }

    \cventry
    {Advisor: Dr. Asako Kanezaki} % Advisor
    {Tokyo Institute of Technology - Dept of Computer Science} % Institution
    {Tokyo, Japan} % Location
    {March 2024 - Sep. 2024} % Date(s)
    {
        \begin{cvitems} % Description(s)
            \item {Offline reinforcement learning for robotic navigation tasks.}
            \item {Development of offline reinforcement learning algorithms (IQL, CQL, OffNav) to improve the learning process from pre-recorded datasets.}
            \item {Use of TSUBAME 4.0 supercomputer for distributed training of deep reinforcement learning models.}
            \item {Training of visual reinforcement learning agents in Vizdoom environment for fast adaptation using SparsE-FeAture-based Regularization for Fine-tuning (SEFAR).}
        \end{cvitems}
    }

    %---------------------------------------------------------
    \cventry
    {Advisor: Dr. Rafael Perez} % Advisor
    {Complutense University of Madrid - Dept of Nanomaterials} % Institution
    {Madrid, Spain} % Location
    {Dec. 2017 - June 2018} % Date(s)
    {
        \begin{cvitems} % Description(s)
            \item {Collaboration with the Nanomagnetism and Magnetization Processes Group at the Institute of Materials Science of Madrid (ICMM).}
            \item {Development of $Fe_{90}Si_{5}B_5$ alloys using voltaic arc melter.}
            \item {Production of microwires by the Taylor-Ulitovsky technique.}
            \item {Characterization of its magnetic properties by Vibrant Sample Magnetometer (VSM) and Kerr Magnetometry techniques.}
            \item {Characterization of its structural properties by X-Ray Diffraction (XRD).}
        \end{cvitems}
    }
\end{cventries}
